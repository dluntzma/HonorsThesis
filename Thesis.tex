\documentclass[prb,preprint]{revtex4-1} 

\usepackage{amsmath}  
\usepackage{amsfonts} 
\usepackage{graphicx} 

\begin{document}


\title{Building the Next Generation Atom Trap Trace Analysis}


\author{Danika Luntz-Martin}
\email{dluntzma@smith.edu}
\affiliation{Department of Physics, Smith College, Northampton, MA 01063}

\author{William Williams}
\email{wwilliams@smith.edu} 
\affiliation{Department of Physics, Smith College, Northampton, MA 01063}

\date{\today}

\begin{abstract}


\end{abstract}


\maketitle 


\section{Introduction} 


\section{Theory}


\section{Methods}


\section{Results}


\section{Analysis}


\section{Discussion}


\section{Conclusion}



\begin{thebibliography}{9}


\bibitem{latexsite} \LaTeX\ Project Web Site, \url{<http://www.latex-project.org/>}.

\bibitem{latexbook}Helmut Kopka and Patrick W. Daly, \textit{A Guide to
\LaTeX}, 4th edition (Addison-Wesley, Boston, 2004).

\bibitem{revtex} REV\TeX\ 4 Home Page, \url{<https://authors.aps.org/revtex4/>}.

\bibitem{cloudLaTeX} On the other hand, you can avoid the installation process
entirely by using a cloud-based \LaTeX\ processor such as ShareLaTeX,
\url{<https://www.sharelatex.com/>}, or write\LaTeX, \url{<https://www.writelatex.com/>}.

\bibitem{mermin} N. David Mermin, ``What's wrong with these equations?,'' 
Phys. Today \textbf{42} (10), 9--11 (1989).  
% Note that the issue number (10) in this citation is required, because
% each issue of Physics Today starts over with page 1.  Also note the use of
% an en-dash (--), not a hyphen (-), for the page range.

\bibitem{feynman} Richard P. Feynman, Robert B. Leighton, and Matthew Sands, 
\textit{The Feynman Lectures on Physics, Vol.\ 1} (Addison-Wesley, 1964), p.~3-10.
% Note that this book is paginated by chapter; "3-10" is a single page reference
% that uses a hyphen, not a range of pages that would us an en-dash (--).


\end{thebibliography}


\end{document}
