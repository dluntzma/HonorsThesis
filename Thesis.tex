\documentclass[prb,preprint]{revtex4-1} 

\usepackage{amsmath}  
\usepackage{amsfonts} 
\usepackage{graphicx} 
\usepackage{color}
\usepackage{ulem}
\usepackage[T1]{fontenc}
\usepackage{mathptmx}
\usepackage{setspace}


\begin{document}
\begin{titlepage}
\thispagestyle{empty}
\begin{center}

\fontsize{20pt}{20pt}\text{   }
\vspace{30mm}

{\fontsize{18pt}{18pt}\selectfont \textbf{The Optical Production of Metastable Krypton Atoms \\ for the Development of the \\ Next Generation of Atom Trap Trace Analysis}}


\vfill
{\fontsize{14pt}{14pt}\selectfont \text{Danika Luntz-Martin}}

\vfill

\singlespacing{
Submitted to the Department of Physics \\
 of Smith College \\
 in partial fulfillment \\
 of the requirements for the degree of \\
 Bachelor of Arts}



\vfill
{\fontsize{14pt}{14pt} \selectfont \text{William Williams,  Honors Project Advisor}}


\vfill
\fontsize{12pt}{12pt}{\today}
\vfill

\end{center}
\end{titlepage}

\pagenumbering{arabic}

\section{Introduction} 
*This sentence should be somewhere in your introduction*  This senior thesis is the first year of a three year project designed to optically produce metastable krypton atoms for purposes of Atom Trap Trace Analysis.

\section{Theory}

Krypton is one of the noble gases; its outer electron shell is completely filled therefore the energy gap between the ground energy state and the first excited state is large. Therefore to laser cool from the ground state would require 123.5nm light which is not currently experimentally possible. To be able to laser cool and trap krypton atoms they need to be in an excited state and that state needs to be metastable so that the atoms cannot decay while we are cooling and trapping them. Krypton has a metastable state that is $~80,000 cm^{-1}$ ($~125 nm$) above the ground state.  This metastable state, which is actually the lowest energy excited state, is doubly forbidden thus requiring three photons to transfer the atom from the ground state to the metastable state.   In order to optically create metastable krypton atoms, we choose to excite the atoms into a higher energy excited state using a two-photon transition at 215nm.  Once in this highly excited state, the atom can emit a photon (the third photon) and decay into the metastable state. 

To get the kryptons to the metastable state we used a 215nm light to drive a two-photon transition from the ground $4p^6$ $^1S_0$ state, to the $5p_{[3/2]_2}$ state, see Figure~\ref{KrEnergyLevels}. Throughout our calculations we call the ground $4p^6$ $^1S_0$ state 'state 1' and the $5p_{[3/2]_2}$ state 'state 2.' There are other states at similar energies to the states in question, but they are not important for our experiment because the transitions are not resonant with our laser light, making it improbable for the atom to transfer to those states. From the $5p_{[3/2]_2}$ state the atom will either decay to an intermediate state and then ground state or it will decay to a metastable state, the $5s_{[3/2]_2}$ state. We called this metastable state 'state 3.' The last possibility is that the atom can ionize either from the the $5p_{[3/2]_2}$ state or the metastable $5s_{[3/2]_2}$ state. 

\begin{figure}[h!]
\centering
\includegraphics[width=6in]{KrEnergyLevels.pdf}
\caption{The energy levels of krypton. This figure only includes energy levels that are relevant to our experiment. Atoms are excited from the $4p^6$ $^1S_0$ state using two-photons to the $5p_{[3/2]_2}$ state then they decay either back to the $4p^6$ $^1S_0$ state or to the $5s_{[3/2]_2}$ state. Atoms can be ionized from both the $5s_{[3/2]_2}$ state and the $5p_{[3/2]_2}$ state.}
\label{KrEnergyLevels}
\end{figure}

Due to the isotope selectivity of laser cooling and trapping, Atom Trap Trace Analysis is able to count a specific isotope chosen by the laser frequency.  Only atoms in the metastable state can be trapped, so the efficiency of the apparatus is limited by the number of atoms transferred to the metastable state. To determine the efficiency of our set up, we calculated the percentage of krypton atoms that ended up in the metastable $5S_{[3/2]_2}$ state using the rate equations.

\begin{equation}
\label{RateEq1}
\frac{dN_1}{dt} = -\omega_{12}N_1 + \frac{1}{\tau_{21}}N_2
\end{equation}

Equation~\ref{RateEq1} gives the rate of change in the number of atoms ($N_1$) in the ground state. $\omega_{12}$ is the rate by which atoms in state 1 (the ground state) which are excited to state 2, the $5P_{[3/2]_2}$ state, via the two-photon transition. The minus sign in front of $\omega_{12}$ indicates the atoms are leaving state 1. $\frac{1}{\tau_{21}}$ is the rate of decay from state 2 back to state 1. Similarly, there are differential equations for the three other states.

\begin{equation}
\label{RateEq2}
\frac{dN_2}{dt} = \omega_{12}N_1 -  \frac{1}{\tau_{21}}N_2 - \frac{1}{\tau_{23}}N_2 - R_2N_2
\end{equation}

\begin{equation}
\label{RateEq3}
\frac{dN_3}{dt} =  \frac{1}{\tau_{23}}N_2 - R_3N_3
\end{equation}

\begin{equation}
\label{RateEq4}
\frac{dN_4}{dt} = R_2N_2 + R_3N_3
\end{equation}

Where $\tau_{23}$ is the decay rate from state 2 to state 3, the metastable state. $R_2$ and $R_3$ are the ionization rates from states 2 and 3 respectively. $N_4$ is the number of atoms which have been ionized, just as $N_2$ and $N_3$ are the number of atoms in state 2 and state 3.

Solving these linearly dependent equations is possible, but the computational time is much shorter if they are put into matrix form.

\begin{equation}
\label{RateEqMatrix}
\begin{bmatrix}
	\frac{dN_1}{dt} \\
	\frac{dN_2}{dt} \\
	\frac{dN_3}{dt} \\
	\frac{dN_4}{dt} \\
\end{bmatrix}
=
\begin{pmatrix}
	-\omega_{12} & \frac{1}{\tau_{21}}  & 0 &  0   \\
	\omega_{12}  & -\frac{1}{\tau_{21}}- \frac{1}{\tau_{23}}-R_2 & 0 & 0 \\
	0  &  \frac{1}{\tau_{23}}  & - R_3 & 0 \\
	0  &  R_2  & R_3 & 0  \\
\end{pmatrix}
\begin{bmatrix}
	N_1 \\
	N_2 \\
	N_3 \\
	N_4 \\
\end{bmatrix}
\end{equation}

The general solution to the first order differential matrix equation is

\begin{equation}
\label{RateEqSol}
N(t) = A_1e^{\lambda_1 t} v_1 + A_2e^{\lambda_2 t} v_2 + A_3e^{\lambda_3 t} v_3 + A_4e^{\lambda_4 t} v_4 
\end{equation}

where $\lambda$ is an eigenvalue of the matrix and $v$ is an eigenvector of the matrix.  A is the amplitude determined by the initial conditions. The initial condition is that all of the atoms are in the ground state, state 1. That is

\begin{equation}
\label{InitialCond}
N_1(0) = 1, \	\	\	\	
N_2(0) = 0, \	\	\	\	
N_3(0) = 0, \	\	\	\	
N_4(0) = 0 
\end{equation}

which allows us to obtain particular solutions to the rate equations. We did all calculations and analysis in Mathematica. Once we had the solutions to the rate equations we needed to define the variables $\omega_{12}$, $\tau_{21}$, $\tau_{23}$, $R_2$ and $R_3$ which make up the eigenvectors and eigenvalues of the matrix.

\subsection{Constant Intensity and Velocity Approximation} 

For our initial calculation we made a couple simplifying approximations. The first approximation we made was to model the intensity of the laser beam profile a step function.  Although the actual laser beam has a gaussian profile, this approximation allows for an exact solution to the above equations.  The height of the step function was constant and set equal to the maximum of the gaussian profile intensity. The width was set to twice the laser waist (also known as the laser radius) of the gaussian laser beam profile.  The second approximation was to assume all of the atoms in the collimated atomic beam were traveling with the root-mean-squared velocity of atomic beams' Maxwell velocity distribution.  The two approximations are standard in the community for evaluating the rate equations.  We later expanded our calculations to account for the gaussian beam profile and the velocity distribution; those calculations can be found in the subsequent sections. 

The decay rates for state 2 ($5P_{[3/2]_2}$) to state 1 (ground state) and state 3 (metastable state) are know quantities.
\begin{equation}
\label{DecayRates} 
\tau_{21} = 1.1x10^7 s, \	\	\	\	 \tau_{23} = 3.1x10^7s
\end{equation} 
\textcolor{blue}{Does NIST have these? I didn't see them cited in the MRI proposal. Yes, here is the citation: Transition probabilities for Kr I lines from wall-stabilized arc measurements, W. E. Ernst and E. Schulz-Gulde, Physica B+C 93, 136–144 (1978)}

The ionization rates are given by 
\begin{equation}
\label{IonizationRates}
R = \sigma_{pi} \frac{I}{\hbar\omega}
\end{equation}

where $I$ is the intensity of the laser calculated from power, $I = \frac{P}{\frac{1}{2}\pi w^2}$ and $w$ is the beam waist. We used a power value of $7.5 W$ based on a conservative estimate from the laser specifications. $\omega$ is the angular frequency of the light defined as $\omega = \frac{2\pi c}{\lambda}$. $\sigma_{pi}$ is the photo-ionization cross section, the probability that a photon will be absorbed by the atom. $\sigma_{pi2} = .5$ Mbarn for the $5p_{[3/2]_2}$ state and $\sigma_{pi2} = .25$ Mbarn for the $5s_{[3/2]_2}$ state.~\cite{Cannon}

The rate by which atoms are excited by two-photon transition from state 1 to state 2 is $\omega_{12}$ given by

\begin{equation}
\label{ExcitationRate}
\omega_{12} = \sigma_0 g \frac{I^2}{(\hbar \omega)^2}
\end{equation}

$\sigma_0$ is the two-photon cross section, the probability that two photons will be absorbed, which is $4.4x10^{-35} cm^{-4}$.~\cite{NIST} $g$ is the on resonance line shape factor, defined as 

\begin{equation}
\label{LineShapeFactor}
g = 2 \sqrt{\frac{Log(2)}{2 \pi (\Delta \omega_L^2 + \Delta \omega_D^2)}}
\end{equation}

with $\Delta\omega_L$ as the line width of the laser. $\Delta \omega_D$ is Doppler line width for the two-photon transition, which is zero in our case because the krypton atoms are moving perpendicular to the laser.  The line shape factor becomes important whenever the line width of the transition is smaller than the line width of the laser.  When this happens, different frequencies of the light have different probabilities of exciting the atom.  The line shape factor, which is a convolution between the gaussian line shape of the laser and the Lorentzian line shape of the atomic transition, takes into account the varying transition probability across the frequency profile of the laser beam. 

The solution to these equations for $7.5 W$ of power inside the build up cavity is given in Figure~\ref{MetaGraph1} which shows the fraction of atoms in the metastable state as a function of the size of the laser's beam waist. The optimum number of atoms end up in the metastable state when the beam waist is close to 20 microns. When the beam waist is smaller than 20 microns more of the atoms are ionized since the intensity is large. When the beam waist is larger than 20 microns, the intensity is too small to drive the two-photon transition leaving most of the atoms in the ground state.

\begin{figure}[h!]
\centering
\includegraphics[width=6in]{MetaGraph1.pdf}
\caption{The fraction of atoms in the metastable state as a function of laser's beam waist. The fraction of atoms reaching the metastable state increases sharply as the beam waist increases until 20 microns. After 20 microns the number of atoms in the metastable state decreases because the intensity is too small.}
\label{MetaGraph1}
\end{figure}

Figure~\ref{TimeGraph} shows the fraction of atoms in each of the states as a function of time for a laser waist of $20 \mu m$, which is helpful to see how the atoms move between states. The maximum time of this graph is the atom-light interaction time given by the width of the step function (twice the laser waist) divided by the root-mean-square velocity of the Maxwell velocity distribution.  The green line is the ground state which starts at one because the initial condition is all of the atoms in the ground state and continuously decreases as time passes. State 1 is the orange line; it quickly increases as atoms are excited through two-photon transition. The number of atoms in state 1 decrease as atoms decay either back to the ground state or to the metastable state. The number of atoms in the metastable state increases as atoms decay to it from the state 1. Atoms only leave the metastable state if they ionize. For this laser waist, only a small fraction of atoms are ionized, see the red line.  It should be noted that the final number of atoms in the metastable state given in Figure~\ref{MetaGraph1} is the number of atoms in state 3 at the end of the atom-light interaction time (the last point on the plot in Figure~\ref{TimeGraph}) plus the fraction of atoms in state 2 that will naturally decay to state 3, about 74 percent.

\begin{figure}[h!]
\centering
\includegraphics[width=6in]{TimeGraph.pdf}
\caption{The fraction of atoms in each state plotted against time. The green line is the ground state. The orange line is the excited $5p_{[3/2]_2}$ state. Blue is the metastable $5s_{[3/2]_2}$ and red is the fraction of atoms that have ionized.}
\label{TimeGraph}
\end{figure}

\subsection{Gaussian Beam Profile}

To make our model more accurate we factored in the intensity profile of the laser beam. From the laser manufacturer's, M Squared, specifications we know that the Ti:Sapphire laser's intensity profile is a two-dimensional gaussian. It was conceptually simple to incorporate the intensity profile into our previous calculation in place of the constant intensity. Computationally, it was not quite as easy.

We replaced our former intensity with by $I = I_0 e^{\frac{-2 x^2}{w^2}}$ where $I_0 = \frac{P}{\frac{1}{2}\pi w^2}$, the same intensity we used as a constant in our previous approximation. However, the addition of the gaussian term means that the equations no longer have analytical solutions. We solved the differential equations numerically.

The resulting graph is noticeably different from our first approximation, see Figure~\ref{AllGraph}. Adding the intensity profile does not change the shape of the peak, but it does make the peak lower and narrower. The location of the peak is now around 15 micrometers, rather than the 20 micrometers from the constant intensity approximation.  The fraction of atoms reaching the metastable state is still quite high having been decreased by just over five percent from roughly .9 to .85.

\begin{figure}[h!]
\centering
\includegraphics[width=6in]{AllGraph.pdf}
\caption{The blue line represents the calculation assuming all atoms have the same velocity (the root-mean-square velocity) and a constant intensity. The green line represents the calculation assuming the root-mean-square velocity and a gaussian intensity profile. Finally, the red line is the calculation using both the gaussian intensity profile and the Maxwell velocity distribution. The root-mean-square velocity is a good approximation; the green and red lines are very close. The constant intensity is not as good an approximation for the more accurate gaussian intensity profile, note the differences between blue line and the green line.}
\label{AllGraph}
\end{figure}


\subsection{Maxwell Velocity Distribution}

Adding the Maxwell velocity distribution presents new difficulties. To account for the range of velocities we first calculated the fraction of atoms in each of a series of intervals. Then we calculated the fraction of atoms from each interval, that is at each different velocity, that reach the metastable state. We then multiplied the fraction of atoms that reach the metastable state from a velocity interval by the fraction of atoms in that velocity interval. Then we added together the resulting fraction from each interval to obtain the total fraction of atoms that end up in the metastable state.  In other words, we evaluated the probability an atom gets transferred to the metastable state for all velocities and then weighted those probabilities according to the Maxwell velocity distribution.  See Figure~\ref{AllGraph} for a comparison of the fraction of atoms reaching the metastable state from each calculation.

Modeling the velocity of the atoms as the root-mean-square velocity is a very good approximation. The differences between using the root-mean-square velocity as a single average value and using the Maxwell velocity distribution were quite small. See the similarity between the green and red lines in Figure~\ref{AllGraph}.  This result gives us confidence that in future calculations we do not have to use the computation time to take into account the full Maxwell distribution.

I don't understand this next paragraph.
The fraction of atoms reaching the metastable state is constant through time. The number of atoms that are in the metastable state at a specific time does change, look back to Figure~\ref{TimeGraph}. The number of atoms in the metastable state increases through time before plateauing. However the total fraction of atoms reaching the metastable is the fraction of atoms that are currently there plus the fraction of atoms in the excited, second state that will decay into the metastable state. The sum of these two numbers remains constant through time which is why our total calculation is stable through time.
  

\section{Titanium-Sapphire Laser}

A titanium-sapphire (Ti-Sapphire) laser uses titanium doped sapphire crystals, hence the name. These crystals are pumped using another laser. We used a \textcolor{blue}{add specific wavelength and type.} Refer to Figure~\ref{LaserSetup} for the experimental set-up of the laser. The pump laser could be operated at various powers which adjusted the overall output of the Ti-Sapphire laser. We operated the pump laser at both 18 W and 12 W.

The first step to operating the laser was to turn on the coolers which cooled the sapphire crystals and the pump laser and then to turn on the pump laser and allow it to warm up and stabilize. The pump laser takes around thirty minutes to warm up, before it is warm it will not generate any light from the Ti-Sapphire laser. It is preferable to allow the pump laser longer to fully stabilize because any drift in the pump laser will cause the Ti-Sapphire deviate from optimum and likely lose lock.

Once the pump laser is warm and stable, the light into the Ti-Sapphire laser must be optimized. The the output from the pump laser passes first through a beam splitter, then through a set of mirrors which can be used to align the beam in all three axis directions. The power to the Ti-Sapphire laser can be measured by using the beam splitter after the Ti-Sapphire laser and reference cavity, again see Figure~\ref{LaserSetup} and a power meter. The alignment in the $z$ direction, the direction of propagation, should not be adjusted, but the horizontal and vertical directions, $x$ and $y$, should both be adjusted to maximize the output power of the Ti-Sapphire laser. 

After optimizing the pump laser into the Ti-Sapphire laser the output voltages can be examined on an oscilloscope and used for diagnostic purposes. The exact setup of the Ti-Sapphire laser is proprietary, however \textcolor{blue}{add basics of TiS here.} The Ti-Sapphire laser from M Squared is controlled from a laptop using the provided software. The desired wavelength is selected on the laptop. After selecting the wavelength, the first step is lock the etalon. The etalon is locked at a point where its error signal crosses zero. The etalon error signal can be seen by selecting 'Etalon Error' from the dropdown menu for one of the monitors and scanning the resonator cavity continuously. The etalon error signal can be seen on the oscilloscope; if it is not symmetric around zero, the pump laser may need to be re-optimized into the Ti-Sapphire laser. A typical signal for the etalon error can be seen in Figure~\ref{EtalonError}. The etalon is then locked by clicking 'Apply Lock' on the computer. The locked signal should be a zero and show only noise. The signal to noise ratio for the etalon is 13.

\begin{figure}[h!]
\centering
\includegraphics[width=6in]{EtalonError.pdf}
\caption{The etalon error signal as seen on the oscilloscope both before, maroon, and after, orange, the etalon has been locked. The signal to noise ratio is 13.}
\label{EtalonError}
\end{figure}

The second step of locking the Ti-Sapphire is the resonator cavity. The resonator cavity narrows  band of wavelengths produced by the laser. The resonator error can be viewed on the oscilloscope by scanning the resonator continuously and selecting the resonator error from the monitor menu. The resonator error must cross zero to support a good lock, if the resonator error does not cross zero it should be adjusted using \textcolor{blue}{look up what exactly what which button this is.} As seen in Figure~\ref{ResonatorError} the signal error ratio is large, 377.

\begin{figure}[h!]
\centering
\includegraphics[width=6in]{ResonatorError.pdf}
\caption{The maroon line is the resonator error signal before it is locked and the orange line is after the resonator has been locked. The signal to noise ratio is 377.}
\label{ResonatorError}
\end{figure}

Next the laser light must be coupled from the Ti-Sapphire laser into the doubler \textcolor{blue}{add technical name}. To optimize the light into the doubler 


\section{Laser Stabilization}

\section{Appendix}

\textcolor{magenta}{Put Mathematica notebooks here!}


\begin{thebibliography}{2}

\bibitem{NIST} A. Kramida, Yu. Ralchenko, J. Reader, J. and NIST ASD Team (2013). NIST Atomic Spectra Database (version 5.1), [Online]. Available: \url{<http://physics.nist.gov/asd>} [Wednesday, 15-Jan-201422:06:55 EST]. National Institute of Standards and Technology, Gaithersburg, MD.

\bibitem{Cannon} B. D. Cannon, ``Model calculations of continuous-wave laser ionization of krypton,'' PNNL-12253 (1999)

\end{thebibliography}


\end{document}
