\documentclass[prb,preprint]{revtex4-1} 

\usepackage{amsmath}  
\usepackage{amsfonts} 
\usepackage{graphicx} 

\begin{document}


\title{Building the Next Generation Atom Trap Trace Analysis}


\author{Danika Luntz-Martin}
\email{dluntzma@smith.edu}
\affiliation{Department of Physics, Smith College, Northampton, MA 01063}

\author{William Williams}
\email{wwilliams@smith.edu} 
\affiliation{Department of Physics, Smith College, Northampton, MA 01063}

\date{\today}

\begin{abstract}


\end{abstract}


\maketitle 


\section{Introduction} 


\section{Theory}

Krypton is one of the noble gases, its outer electron shell is completely filled, as such the energy gap between the ground energy state and the first excited state is large (XXXX). 

To be able to trap krypton atoms they need to be in a metastable state. To get the kryptons to the metastable state we used a 215nm light to drive a two-photon transition from the ground state, through the ZZZZZZ state to the XXXXXX state. Throughout our calculations we call the ground state 'state 1' and the XXXXXX state 'state 2.' The ZZZZZ state is not important for our experiment; it is just a intermediate state between the ground state and the XXXXXX state. From the XXXXX state the atom will either decay to the ground state (through the ZZZZZ state) or it will decay to a metastable state, the YYYYYY state. We called this metastable state 'state 3.' The last possibility is that the atom can ionize either from the the XXXXX state or the metastable YYYYY state.

To determine the efficiency of our set up, we needed to calculate the percentage of krypton atoms that ended up in the 

\section{Methods}


\section{Results}


\section{Analysis}


\section{Discussion}


\section{Conclusion}



\begin{thebibliography}{0}



\end{thebibliography}


\end{document}
